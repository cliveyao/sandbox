\chapter{Introduction}
The first chapter of a \lq well-structured\rq\ report is always an
introduction, setting the scene with motivation and context (as in
Sec.~\ref{intro}) and then looking ahead to summarise what's in the
rest of the report (as in Sec.~\ref{intro:contents}). It's the
bit that readers look at first --- {\em so make sure it hooks them!}
%
\section{Context and motivation}\label{intro}
This is a template that shows an overall structure for the
printed document, and shows how to construct it with a master file
(\texttt{report.tex}) plus subsidiary files (\texttt{chap1.tex},
\dots, \texttt{app1.tex}, \dots, \texttt{biblio.tex}).
\par
Comparison .... FAQs: \Quote{How can I do \dots\ in LaTeX}.
\par
However, this is {\em not} a textbook Latex. It is at \textsl{Wikipedia}
\cite{WL} includes the \nss\ material and is good for reference too.
\par
For more advanced features see \eg\ blabla.
\par
Advice for report-writing is available\footnote{From  \texttt{bob.johnson@dur.ac.uk}}
in room CM315.
%
\section{Contents}\label{intro:contents}
The main body of this report is divided as follows.
\par
Chap.~\ref{sec:graphics} deals with graphics and includes
Sec.~\ref{sec:tables} about tables. The Conclusion, in
Chap.~\ref{andfinally}, summarises what's been achieved, the open
questions and what could be done next.
\par
Then comes the Bibliography, listing all sources of material, data and
computer programs used, \etc. Its construction is explained in
\cite[Sec.~4.2]{NSS} and there's more about it in App.~\ref{app:refs}.
%
