\chapter{Using a PC}\label{app:pc}
You may want \LaTeX\ on your own computer.
\par
A popular version of \TeX/\LaTeX\ for a Windows PC is \Quote{MikTex}
\cite{MKT}. \miktex\ is free, and is available online for 
download \cite{LAT} or locally on a DVD from
\texttt{bob.johnson@dur.ac.uk}, room CM315. It's also installed on the
ITS Networked PC Service under \texttt{Programs | Miscellaneous}.
\par
To use \miktex\ easily you need a dedicated
editor or IDE\footnote{Computer scientists say \lq Integrated 
Development Environment'.}. A good one is \Quote{WinEdt} \cite{WDT}, 
which runs under all recent versions of Microsoft Windows and integrates 
well with \miktex. \textsl{WinEdt} is free for 31 days; to use
it thereafter costs about \$40.
\par 
Completely free rivals include \Quote{TeXnicCenter} \cite{TXC}, which is
provided on the \miktex\ DVD and also set up on the ITS Networked PC
Service under \texttt{Programs | Miscellaneous | MikTeX}.
\par
Others include \eg \Quote{WinShell} \cite{WSH}, but are untried.
\par
All these editors/IDEs have a familiar style of graphical user-interface
with a toolbar and pull-down menus for all the common tasks involved in
editing source files, running \LaTeX\ and viewing the results.
